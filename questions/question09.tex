\section{Задание линейных отображений (операторов) матрицами. Изменение матрицы линейного отображения при переходе к другим базисам. Нахождение ядра и образа при помощи матрицы}

Пусть $\A: V \to W$ --- линейное отображение, $e_1, \ldots, e_m$ --- базис в $V$, а $f_1, \ldots, f_n$ --- базис в $W$.

\begin{definition}
    \textit{Матрицей линейного отображения} $\A: V \to W$ по отношению к базисам $e_1, \ldots, e_m$ и $f_1, \ldots, f_n$ называется матрица
    $
    A = 
    \begin{pmatrix}
        a^1_1 & \cdots & a^1_m\\
        \vdots & \ddots & \vdots\\
        a^n_1 & \cdots & a^n_m
    \end{pmatrix}
    $
    размера $n \times m$, в которой $i$-ый столбец состоит из координат вектора $\A(e_i)$ в базисе $f_1, \ldots, f_n$: $\A e_i = a^j_if_j$.
\end{definition}

\begin{proposal}
    Пусть $x = x^je_j$ --- произвольный вектор из $V$, а $y = y^if_i$ --- его образ в $W$, т.\,е. $y = \A x$. Тогда $y^i = a^i_jx^j$.
\end{proposal}

\begin{proof}
    Действительно, $y^if_i = y = \A x = \A(x^je_j) = x^j\A e_j = x^ja^i_jf_i$. Т.\,к. $\{f_i\}_{i = 1}^n$ --- базис, отсюда следует, что $y^i = a^i_jx^j$.
\end{proof}

\begin{proposal}
    Пусть $\dim V = m$, $\dim W = n$. Тогда $\Hom_\F(V, W) \simeq \Mat_\F(n, m)$.
\end{proposal}

\begin{proof}
    Выберем базисы $e_1, \ldots, e_m$ и $f_1, \ldots, f_n$ в $V$ и $W$ соответственно. Определим отображение $\Hom_\F(V, W) \to \Mat_\F(n, m)$, которое сопоставляет линейному отображению его матрицу в выбранных базисах. Непосредственно проверяется, что это отображение линейно. Кроме того, оно биективно: обратное отображение сопоставляет матрице $\underset{n \times m}{A} = (a^i_j)$ линейного отображения, определяется в координатах формулой из предыдущего предложения. Следовательно, такое отображение $\Hom_\F(V, W) \to \Mat_\F(n, m)$ является изоморфизмом.
\end{proof}

\begin{theorem}[Закон изменения матрицы линейного отображения]
    Имеет место соотношение $A^\prime = D^{-1}AC$, где $A$ --- матрица линейного отображения $\A: V \to W$ по отношению к базисам $e_1, \ldots, e_m$ и $f_1, \ldots, f_n$; $A^\prime$ --- матрица линейного отображения $\A$ по отношению к базисам $e_{1^\prime}, \ldots, e_{m^\prime}$ и $f_{1^\prime}, \ldots, f_{n^\prime}$; $C = C_{e \to e^\prime}$ --- матрица перехода от $e_1, \ldots, e_m$ к $e_{m^\prime}, \ldots, e_{m^\prime}$; $D = D_{f \to f^\prime}$ --- матрица перехода от $f_1, \ldots, f_n$ к $f_{1^\prime}, \ldots, f_{n^\prime}$.
\end{theorem}

\begin{proof}
    Пусть $C = (c^i_{i^\prime})$, $A = (a^j_i)$, тогда $\A e_{i^\prime} = \A(c^i_{i^\prime}e_i) = c^i_{i^\prime}\A e_i = c^i_{i^\prime}a^j_if_j$. С другой стороны, если $A^\prime = (a^{j^\prime}_{i^\prime})$ и $D = (d^j_{j^\prime})$, то $a^{j^\prime}_{i^\prime}f_{j^\prime} = a^{j^\prime}_{i^\prime}d^j_{j^\prime}f_j$. Сравнивая два последних соотношения с учётом того, что ${f_j}_{j = 1}^n$ --- базис, получаем $a^j_ic^i_{i^\prime} = d^j_{j^\prime}a^{j^\prime}_{i^\prime}$. В матричном виде это эквивалентно $AC = DA^\prime \Rightarrow A^\prime = D^{-1}AC$.
\end{proof}

\textbf{Поиск ядра и образа линейного оператора по его матрице}. Пусть имеем матрицу $A$ оператора $\A$ в каком-то базисе. Приведя её к ступенчатому виду, сможем найти базис системы столбцов матрицы $A$ (его будут составлять столбцы, в которых есть лидеры). Вспомним, что по столбцам $A$ написаны образы базисных векторов при отображении $\A$, а мы нашли базис этой системы. Это значит, что найденные нами столбцы есть базис $\Im\A$.

$\Ker\A$ --- это просто пространство решений СЛУ $Ax = 0$. Чтобы найти базис ядра, нам нужно просто найти её ФСР.

