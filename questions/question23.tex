\section{Существование жордановой нормальной формы матрицы над алгебраически замкнутым полем}

Посмотрим, как выглядит матрица оператора $\A\big|_{R_\lambda}$ (ограничения оператора $\A$ на корневое подпространства $R_\lambda$). Т.\,к. $(\A - \lambda \cdot \id)\big|_{R_\lambda}$ является нильпотентным оператором, в пространстве $R_\lambda$ можно выбрать нормальный базис для этого оператора. Тогда матрица оператора $(\A - \lambda \cdot \id)\big|_{R_\lambda}$ в этом базисе будет состоять из блоков вида
$
\begin{pmatrix}
    0 & 1 &  &  \\
      & 0 & \ddots & \\
      & & \ddots & 1\\
      & & & 0
\end{pmatrix}
$. А значит, матрица оператора $\A\big|_{R_\lambda}$ состоит из блоков вида
$
\begin{pmatrix}
    \lambda & 1 &  &  \\
      & \lambda & \ddots & \\
      & & \ddots & 1\\
      & & & \lambda
\end{pmatrix}
$, т.\,е. жордановых клеток.

\begin{theorem}
    Для любого оператора $\A$ в пространстве $V$ над алгебраически замкнутым полем существует жорданов базис (в котором оператор имеет жорданову нормальную форму).
\end{theorem}

\begin{proof}
    Существование жордановой формы является прямым следствием теорем о разложении в прямую сумму корневых подпространств и существовании нормального вида для нильпотентных операторов. Действительно, пусть $\lambda_1, \ldots, \lambda_k$ --- все собственные значения $\A$. Выберем в каждом корневом подпространстве $R_{\lambda_i}$ нормальный базис для нильпотентного оператора $(\A - \lambda_i \cdot \id)\big|_{R_{\lambda_i}}$. Тогда объединение этих базисов даст жорданов базис для оператора $\A$ в силу наличия корневого разложения $V = R_{\lambda_1} \oplus \ldots \oplus R_{\lambda_k}$ (здесь мы пользуемся алгебраической замкнутостью поля).
\end{proof}

