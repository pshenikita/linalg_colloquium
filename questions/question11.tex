\section{Векторное пространство линейных отображений. Алгебра линейных операторов. Изоморфизм алгебры матриц и алгебры линейных операторов}

Утверждение, которое доказывалось в первом семестре:

\begin{proposal}
    Пусть $U \overset{\psi}{\to} V \overset{\varphi}{\to} W$. Если $\varphi$, $\psi$ --- линейные, то $\varphi \circ \psi$ линейное.
\end{proposal}

\begin{definition}
    Множество $A$ с операциями сложения, умножения на скаляры и умножения элементов $A$, удовлетворяющее аксиомам:
    \begin{enumerate}[nolistsep]
        \item $(A, +)$ --- ассоциативное кольцо;
        \item $A$ --- векторное пространство (относительно операций сложения и умножения на скаляры);
        \item $\lambda(ab) = (\lambda a)b$ $\forall \lambda \in \F$, $\forall a, b \in A$
    \end{enumerate}
    называеся \textit{линейной алгеброй}.
\end{definition}

\begin{theorem}
    Множество всех линейных операторов на $V$ с операциями сложения и умножения на скаляры и композиции является линейной алгеброй. Причём, это алгебра изоморфна алгебре матриц над тем же полем.
\end{theorem}

\begin{proof}
    Первые две аксиомы уже проверены. Третья проверяется непосредственно:
    \[
        (\lambda(ab))(v) = \lambda((ab)(v)) = \lambda a(b(v)) \Rightarrow \lambda(ab) = (\lambda a)b.
    \]

    Вторая часть утверждения очевидна --- мотивировкой именно такого определения произведения матриц было как раз то, что это будет матрица композиции линейных отображений.
\end{proof}

