\section{Вычисление собственных значений и собственных векторов с помощью матрицы. Характеристический многочлен}

\begin{definition}
    Многочлен $\chi_\A(t) \vcentcolon = \det(\A - t \cdot \id)$ называется \textit{характеристическим многочленом} оператора $\A$.
\end{definition}

Так как характеристический многочлен определяется как детерминант оператора, его можно вычислять как определитель матрицы этого оператора в любом базисе:
\[
    \chi_\A(t) = \det(A - t \cdot E) = \det
    \begin{pmatrix}
        a^1_1 - t & a^1_2 & \cdots & a^1_n\\
        a^2_1 & a^2_2 - t & \cdots & a^2_n\\
        \vdots & \vdots & \ddots & \vdots\\
        a^n_1 & a^n_2 & \cdots & a^n_n - t\\
    \end{pmatrix}.
\]

Из этой формулы ясно, что $\deg\chi_\A(t) = \dim V$, кроме того, при $\chi_\A(t) = p_nt^n + p_{n - 1}t^{n - 1} + \ldots + p_0$, имеем $p_n = (-1)^n$, $p_{n - 1} = (-1)^{n - 1}\tr\A$, $p_0 = \det\A$.

\begin{proposal}
    Собственные значения оператора $\A$ --- это в точности корни его характеристического многочлена.
\end{proposal}

\begin{proof}
    Если $\lambda$ --- собственное значение, то оператор $\A - \lambda \cdot \id$ вырожден, т.\,е. $\det(\A - \lambda \cdot \id) = 0$, а значит, $\lambda$ --- корень многочлена $\chi_\A(t)$. Обратно, если $\chi_\A(\lambda) = 0$, то $\det(\A - \lambda \cdot \id) = 0$. Поэтому $\Ker(\A - \lambda \cdot \id) \ne \{\bs{0}\}$, а значит, $\lambda$ --- собственное значение.
\end{proof}

\textbf{Поиск собственных векторов}. Сначала находим корни характеристического многочлена, т.\,е. собственные значения. Теперь найдём подпространство $V_\lambda$ для собственного значения $\lambda$. Иными словами, хотим найти все такие векторы, что $\A v = \lambda v$. Это равносильно системе уравнений с матрицей $(A - \lambda E)$. Она и задаёт искомое подпространство.
