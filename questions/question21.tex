\section{Жордановы клетки и матрицы, их характеристические и минимальные многочлены. Жорданов базис}

\begin{definition}
    Матрица вида
    $
    J_\lambda =
    \begin{pmatrix}
        \lambda & 1 & & \\
         & \lambda & \ddots & \\
         & & \ddots & 1 \\
         & & & \lambda
    \end{pmatrix}
    $ называется \textit{жордановой клеткой}. Если матрица оператора $\A$ в некотором базисе является блочно-диагональной с блоками вида $J_\lambda$ (возможно, соответствующих различным $\lambda$), то такая матрица называется \textit{жордановой нормальной формой} оператора $\A$. Базис, в котором оператор имеет жорданову нормальную форму, называется \textit{жордановым}.
\end{definition}

\begin{proposal}
    Минимальный аннулирующий многочлен оператора $\A$ над полем $\C$ есть $P(t) = \prod\limits_{i = 1}^k(t - \lambda_i)^{m_i}$, где $\lambda_1, \ldots, \lambda_k$ --- все собственные значения $\A$, а $m_i$ --- размер максимальной жордановой клетки, отвечающей $\lambda_i$.
\end{proposal}

\begin{proof}
    Мы имеет $R_{\lambda_i} = \Ker(\A - \lambda_i \cdot \id)^{m_i}$, поэтому многочлен $(t - \lambda_i)^{m_i}$ является минимальным аннулирующим для оператора $\A\big|_{R_{\lambda_i}}$. В силу теоремы о корневом разложении, $\forall v \in V$ представляется в виде $v = \sum\limits_iv_i$, где $v_i \in R_{\lambda_i}$. Т.\,к. $P(\A)$ содержит множитель $(\A - \lambda_i \cdot \id)^{m_i}$, мы имеем $P(\A)v_i = \bs{0}$, т.\,е. $P(\A)v = \bs{0}$, и многочлен $P(t)$ аннулирует оператор $\A$. С другой стороны, любой многочлен $Q(t)$, аннулирующий оператор $\A$, делится на минимальный аннулирующий многочлен для оператора $\A\big|_{R_{\lambda_i}}$, т.\,е. на $(t - \lambda_i)^{m_i}$, для каждого $\lambda_i$. Следовательно, $Q(t)$ делится на $P(t)$, и $P(t)$ --- минимальный многочлен.
\end{proof}

Про то, как искать жорданов базис, лучше всего \href{http://halgebra.math.msu.su/staff/klyachko/teaching/lin.al/JB1.PDF}{написал} Антон Александрович Клячко. Однако там этот алгоритм описан очень неподробно, когда-нибудь он появится здесь со всеми обоснованиями и доказательствами. Также напишу, как всё-таки искать количество жордановых базисов над конечным полем. То, что Антон Александрович записывает клетки с единицами под собственными значениями, а не над, как это принято, \textbf{существенно}. 

Как нетрудно заметить, у жордановой клетки минимальный многочлен совпадает с характеристическим. На самом деле, такие матрицы обладают особенным свойством.

\begin{problem}[А.\,А. Клячко]
    Докажите, что матрица коммутирует только с многочленами от себя тогда и только тогда, когда её минимальный многочлен совпадает с характеристическим.
\end{problem}

Нам понадобятся три очень технические леммы.

\begin{lemma}[Костя Зюбин]
    Пусть у жордановой матрицы $A$ минимальный многочлен совпадает с характеристическим. Тогда диагональ $A$ является многочленом от неё.
\end{lemma}

\begin{proof}
    Т.\,к. минимальный многочлен совпал с характеристическим, то каждому уникальному собственному значению $\lambda_i$ соответствует ровно одна клетка (из предложения 1), т.\,е. матрица выглядит вот так:
    \[
        A =
        \begin{pmatrix}
            J_1 & & \\
             & \ddots & \\
             & & J_m
        \end{pmatrix},
    \]
    где $J_i$ --- жорданова клетка размером $m_i$ с собственным значением $\lambda_i$, причём все такие $\lambda_i$ попарно различны. Рассмотрим многочлен $\ds g_i(t) \vcentcolon = \frac{\chi_A(t)}{(t - \lambda_i)^{m_i}}$. Этот многочлен, как нетрудно заметить, будет аннулировать все клетки, кроме $J_i$. Это значит, что
    \[
        g_i(A) =
        \begin{pmatrix}
            g_i(J_i) & \\
             & \scalebox{2}{$0$}
        \end{pmatrix}
    \]
    (условно будем рисовать клетку в верхнем левом углу). Пусть $\mu_{G_i}(t)$ --- минимальный многочлен матрицы $G_i = g_i(J_i)$ и $\lambda_i \ne 0$ (мы потом отдельно рассмотрим случай вырожденных клеток). Тогда свободный член $\mu_{G_i}(t)$ не равен нулю. Действительно, в противном случае $\mu_{G_i}(t) = q_{G_i}(t) \cdot t \Rightarrow q_{G_i}(G_i)G_i = 0 \Rightarrow q_{G_i}(G_i) = 0 G_i^{-1} = 0$, причём $\deg q_{G_i} < \deg \mu_{G_i}$ --- противоречие. Теперь будем строить искомый многочлен в несколько шагов:
    \[
        \mu_{G_i}(g_i(A)) =
        \left(
        \begin{array}{c | c}
            0 & \\
            \hline
             & \mu_{G_i}(0)E
        \end{array}
        \right)
    \]
    \[
        \mu_{G_i}(g_i(A)) - \mu_{G_i}(0)E =
        \left(
        \begin{array}{c | c}
            -\mu_{G_i}(0)E & \\
            \hline
             & 0
        \end{array}
        \right)
    \]
    \[
        \lambda_iE - \frac{\lambda_i}{\mu_{G_i}(0)}\mu_{G_i}(g_i(A)) =
        \left(
        \begin{array}{c | c}
            \lambda_iE & \\
            \hline
             & 0
        \end{array}
        \right).
    \]

    И вот отсюда
    \[
        \br{\sum_i\lambda_i}E - \sum_{\lambda_i \ne 0}\frac{\lambda_i}{\mu_{G_i}(0)} \cdot \mu_{G_i}(g_i(A)) =
        \begin{pmatrix}
            \fbox{$\lambda_1E$} & & & \\
             & \fbox{$\lambda_2E$} & &\\
             & & \ddots &\\
             & & & \fbox{$\lambda_mE$}
        \end{pmatrix}.
    \]

    А теперь заметим, что при $\lambda_j = 0$ в получившейся матрице $j$-ый блок будет нулевой матрицей (там будет стоять пустая сумма), а ровно этого и хотелось. Так что найденный нами многочлен переводит $A$ в её диагональ.
\end{proof}

\begin{lemma}
    С матрицей $D = \operatorname{diag}(\overbrace{\lambda_1, \ldots, \lambda_1}^{m_1}, \overbrace{\lambda_2, \ldots, \lambda_2}^{m_2}, \cdots, \overbrace{\lambda_k, \ldots, \lambda_k}^{m_k})$ коммутируют такие и только такие матрицы:
    \[
        \begin{pmatrix}
            B_1 & & \\
             & \ddots & \\
             & & B_k
        \end{pmatrix},
    \]
    где $B_i$ --- блоки размера $m_i$.
\end{lemma}

\begin{proof}
    То, что такие коммутируют, очевидно (проверяется непосредственно). Докажем, что коммутируют только такие. Пусть $B$ коммутирует с $D$. Тогда умножив $D$ на $B$ слева, мы каждую строку $B$ домножим на соответствующий коэффициент. А умножив $D$ на $B$ справа, мы каждый столбец умножим на соответствующий коэффициент. Итак, получаем:
    \begin{multline*}
        \footnotesize
        \begin{pmatrix}
            \lambda_1b_{11} & \ldots & \lambda_1b_{1m_1} & \lambda_1b_{1, m_1 + 1} & \ldots & \lambda_1b_{1, m_1 + m_2} & \ldots & \lambda_1b_{1n}\\
            \vdots & \ddots & \vdots & \vdots & \ddots & \vdots & \ddots & \vdots\\
            \lambda_1b_{m_11} & \ldots & \lambda_1b_{m_1m_1} & \lambda_1b_{m_1, m_1 + 1} & \ldots & \lambda_1b_{m_1, m_1 + m_2} & \ldots & \lambda_1b_{m_1n}\\
            \lambda_2b_{m_1 + 1, 1} & \ldots & \lambda_2b_{m_1 + 1, m_1} & \lambda_2b_{m_1 + 1, m_1 + 1} & \ldots & \lambda_2b_{m_1 + 1, m_1 + m_2} & \ldots & \lambda_2b_{m_1 + 1, n}\\
            \vdots & \ddots & \vdots & \vdots & \ddots & \vdots & \ddots & \vdots\\
            \lambda_2b_{m_1 + m_2, 1} & \ldots & \lambda_2b_{m_1 + m_2, m_1} & \lambda_2b_{m_1 + m_2, m_1 + 1} & \ldots & \lambda_2b_{m_1 + m_2, m_1 + m_2} & \ldots & \lambda_2b_{m_1 + m_2, n}\\
            \vdots & \ddots & \vdots & \vdots & \ddots & \vdots & \ddots & \vdots\\
            \lambda_kb_{n1} & \ldots & \lambda_kb_{nm_1} & \lambda_kb_{n, m_1 + 1} & \ldots & \lambda_kb_{n, m_1 + m_2} & \ldots & \lambda_kb_{nn}\\
        \end{pmatrix} =\\ =\footnotesize
        \begin{pmatrix}
            \lambda_1b_{11} & \ldots & \lambda_1b_{1m_1} & \lambda_2b_{1, m_1 + 1} & \ldots & \lambda_2b_{1, m_1 + m_2} & \ldots & \lambda_kb_{1n}\\
            \vdots & \ddots & \vdots & \vdots & \ddots & \vdots & \ddots & \vdots\\
            \lambda_1b_{m_11} & \ldots & \lambda_1b_{m_1m_1} & \lambda_2b_{m_1, m_1 + 1} & \ldots & \lambda_2b_{m_1, m_1 + m_2} & \ldots & \lambda_kb_{m_1n}\\
            \lambda_1b_{m_1 + 1, 1} & \ldots & \lambda_1b_{m_1 + 1, m_1} & \lambda_2b_{m_1 + 1, m_1 + 1} & \ldots & \lambda_2b_{m_1 + 1, m_1 + m_2} & \ldots & \lambda_kb_{m_1 + 1, n}\\
            \vdots & \ddots & \vdots & \vdots & \ddots & \vdots & \ddots & \vdots\\
            \lambda_1b_{m_1 + m_2, 1} & \ldots & \lambda_1b_{m_1 + m_2, m_1} & \lambda_2b_{m_1 + m_2, m_1 + 1} & \ldots & \lambda_2b_{m_1 + m_2, m_1 + m_2} & \ldots & \lambda_kb_{m_1 + m_2, n}\\
            \vdots & \ddots & \vdots & \vdots & \ddots & \vdots & \ddots & \vdots\\
            \lambda_1b_{n1} & \ldots & \lambda_1b_{nm_1} & \lambda_2b_{n, m_1 + 1} & \ldots & \lambda_2b_{n, m_1 + m_2} & \ldots & \lambda_kb_{nn}\\
        \end{pmatrix}.
    \end{multline*}

    Осталось приравнять элементы на соответствующих местах. Взглянем на левые верхние подматрицы $m_1 \times m_1$. Они совпадают у обеих матриц. Затем взглянем на подматрицу $M^{i = m_1 + 1, \ldots, m_1 + m_2}_{j = m_1 + 1, \ldots, m_1 + m_2}$. Они тоже совпадают! И так далее. Причём, как нетрудно увидеть, совпадают в этих матрицах только эти блоки. Получили то, что хотели.
\end{proof}

\begin{lemma}
    С жордановыми клетками коммутируют только многочлены от них\footnotemark.
\end{lemma}

\footnotetext{Имеется в виду <<они и только они>>. Однако то, что матрица коммутирует с многочленами от себя, очевидно. Нам интересен именно тот случай, когда других коммутирующих нет.}

\begin{proof}
    Давайте проверим. Пусть
    $
    J_\lambda =
    \begin{pmatrix}
        \lambda & 1 & & 0\\
         & \lambda & \ddots & \\
         & & \ddots & 1 \\
        0 & & & \lambda
    \end{pmatrix}
    $ коммутирует с матрицей $B = (b^i_j)$. Перемножим их с обеих сторон:
    \[
    \begin{pmatrix}
        \lambda & 1 & & \\
         & \lambda & \ddots & \\
         & & \ddots & 1 \\
         & & & \lambda
    \end{pmatrix} \cdot
    \begin{pmatrix}
        b_{11} & b_{12} & \ldots & b_{1n}\\
        b_{21} & b_{22} & \ldots & b_{2n}\\
        \vdots & \vdots & \ddots & \vdots\\
        b_{n1} & b_{n2} & \ldots & b_{nn}\\
    \end{pmatrix} =
    \begin{pmatrix}
        \lambda b_{11} + b_{21} & \lambda b_{12} + b_{22} & \ldots & \lambda b_{1n} + b_{2n}\\
        \lambda b_{21} + b_{31} & \lambda b_{22} + b_{32} & \ldots & \lambda b_{2n} + b_{3n}\\
        \vdots & \vdots & \ddots & \vdots\\
        \lambda b_{n - 1, 1} + b_{n1} & \lambda b_{n - 1, 2} + b_{n2} & \ldots & \lambda b_{n - 1, n} + b_{nn}\\
        \lambda b_{n1} & \lambda b_{n2} & \ldots & \lambda b_{nn}\\
    \end{pmatrix}
    \]
    \[
    \begin{pmatrix}
        b_{11} & b_{12} & \ldots & b_{1n}\\
        b_{21} & b_{22} & \ldots & b_{2n}\\
        \vdots & \vdots & \ddots & \vdots\\
        b_{n1} & b_{n2} & \ldots & b_{nn}\\
    \end{pmatrix} \cdot
    \begin{pmatrix}
        \lambda & 1 & & \\
         & \lambda & \ddots & \\
         & & \ddots & 1 \\
         & & & \lambda
    \end{pmatrix} =
    \begin{pmatrix}
        \lambda b_{11} & b_{11} + \lambda b_{12} & \ldots & b_{1, n - 1} + \lambda b_{1n}\\
        \lambda b_{21} & b_{21} + \lambda b_{22} & \ldots & b_{2, n - 1} + \lambda b_{2n}\\
        \vdots & \vdots & \ddots & \vdots\\
        \lambda b_{n1} & b_{n1} + \lambda b_{n2} & \ldots & b_{n, n - 1} + \lambda b_{nn}\\
    \end{pmatrix}.
    \]

    Теперь приравняем соответствующие элементы получившихся матриц. Из первого столбца, $b_{i1} = 0$, $i = 2, \ldots, n$. Теперь, из второго столбца, $b_{i2} = 0$, $i = 3, \ldots, n$. И так далее. Получаем, что матрица $B$ верхнетреугольная. Но это не всё, что можно про неё сказать. Приравняем клетки с координатами $(1, 2)$, получим $b_{11} + \lambda b_{12} = \lambda b_{12} + b_{22} \Rightarrow b_{11} = b_{22}$. Теперь, приравняв клетки с координатами $(2, 3)$, аналогично получим $b_{22} = b_{33}$. И так далее. Получаем, что $b_{11} = \ldots = b_{nn}$. Это мы приравняли диагональ на одну выше главной. Потом приравниваем диагонали ещё на одну выше, и ещё, и т.\,д. Получаем, что матрица $B$ имеет вид:
    \[
        B =
        \begin{pmatrix}
            b_0 & b_1 & \ddots & b_{n - 2} & b_{n - 1}\\
                & b_0 & b_1 & \ddots & b_{n - 2}\\
                &     & b_0    & \ddots & \ddots\\
                &     &        & \ddots & b_1\\
                &     &        &        & b_0\\
        \end{pmatrix}.
    \]
    Или же, $B = b_0 + b_1N + b_2N^2 + \ldots + b_{n - 1}N^{n - 1}$, где $N = A - \lambda E$ --- нильпотентный оператор. А т.\,к. $B$ --- многочлен от $A - \lambda E$, то это и многочлен от $A$ (подставить, раскрыть скобки).
\end{proof}

\begin{solution}
    $\Leftarrow$. Пусть $\chi_A(t) = \mu_A(t)$ и $BA = AB$. По лемме 1, диагональ $A$ --- это некоторый многочлен от $A$, поэтому $B$ коммутирует и с диагональю $A$. А значит, по лемме 2, матрица $B$ имеет блочно-диагональный вид с блоками, равными по размерам жордановым клеткам матрицы $A$. Коммутирование такой матрицы с жордановой матрицей $A$ равносильно коммутированию соответствующих блоков матрицы $B$ с жордановыми клетками $A$. Значит, матрица $B$ имеет вид
    \[
        B =
        \begin{pmatrix}
            g_1(J_1) & & \\
             & \ddots & \\
             & & g_m(J_m)
        \end{pmatrix},
    \]
    где $g_i$ --- многочлены. Осталось из этих многочленов, которые жордановы клетки $A$ переводит в блоки $B$ собрать единый многочлен $f$, который всю матрицу $A$ переведёт во всю матрицу $B$. Для каждого $i$ положим $p_i(t) = \prod\limits_{j \ne i}(t - \lambda_j)^{m_j}$. Т.\,к. все $\lambda_i$ различны, $\gcd(p_i, (x - \lambda_i)^{m_i}) = 1$. Поэтому существуют такие многочлены $a_i$ и $b_i$, что $a_i(t)p_i(t) + b_i(t)(t - \lambda_i)^{m_i} = 1$. Отсюда $(a_ip_i)(J_{j})$ равняется $\underset{m_i \times m_i}{E}$ при $j = i$ и $0$ при $j \ne i$. Отсюда
    \[
        (a_ip_ig_i)(A) =
        \begin{pmatrix}
            (a_ip_ig_i)(J_1) & & \\
             & \ddots & \\
             & & (a_ip_ig_i)(J_m)
        \end{pmatrix} = 
        \begin{pmatrix}
            0 & & \\
            & E \cdot g_i(J_i) & \\
             & & 0\\
        \end{pmatrix} =
        \begin{pmatrix}
            0 & & \\
            & g_i(J_i) & \\
             & & 0\\
        \end{pmatrix}
    \]
    где $g_i(J_i)$ стоит на месте $i$-го блока. Наконец, $B = \sum\limits_{i = 1}^r(a_ip_ig_i)(A)$.

    $\Rightarrow$. Пусть у матрицы $A$ характеристический многочлен не равен минимальному. Тогда у неё какому-то собственному значению $\lambda$ соответствуют (хотя бы) две клетки. Понятно, что без ограничения общности можно рассматривать случай
    $
    J =
    \begin{pmatrix}
        J_1 & \\
         & J_2
    \end{pmatrix}
    $, где $J$ --- жорданова форма матрицы $A$, а $J_1$ и $J_2$ --- жордановы клетки размера $m_1$ и $m_2$ соответственно с собственным значением $\lambda$ на диагонали. Пусть $J = P^{-1}AP$ и $D = \operatorname{diag}(\underbrace{1, \ldots, 1}_{m_1}, \underbrace{0, \ldots, 0}_{m_2})$. Тогда
    \[
        PDP^{-1} \cdot A = PDJP^{-1},\qquad A \cdot PDP^{-1} = PJDP^{-1}.
    \]

    А по лемме 2 матрицы $J$ и $D$ коммутируют, т.\,е. $DJ = JD$. Отсюда $PDP^{-1} \cdot A = A \cdot PDP^{-1}$, т.\,е. $A$ и $PDP^{-1}$ коммутируют. Предположим, что $PDP^{-1} = f(A)$, где $f$ --- многочлен. Тогда
    \[
        \begin{pmatrix}
            1 &  &  &  \\
             & \ddots &  & \\
             & & 1 & \\
             & & & \bigzero\\
        \end{pmatrix} =
        D = P^{-1}f(A)P = f(P^{-1}AP) = f(J).
    \]

    Теперь воспользуемся результатом задачи 1202 из <<Сборника задач по аналитической геометрии и линейной алгебре>> под ред. Смирнова. Утверждение этой задачи заключается в том, что
    \[
        f(J_i) =
        \begin{pmatrix}
            \ds f(\lambda) & \ds\frac{f^\prime(\lambda)}{1!} & \ds\frac{f^{\prime\prime}(\lambda)}{2!} & \ds\frac{f^{\prime\prime\prime}(\lambda)}{3!} & \ldots & \ds\frac{f^{(n - 1)}(\lambda)}{(n - 1)!}\\
            0 & \ds f(\lambda) & \ds\frac{f^\prime(\lambda)}{1!} & \ds\frac{f^{\prime\prime}(\lambda)}{2!} & \ldots & \ds\frac{f^{(n - 2)}(\lambda)}{(n - 2)!}\\
            \vdots & \vdots & \vdots & \vdots & \vdots & \vdots\\
            0 & 0 & 0 & 0 & \ldots & f(\lambda)
        \end{pmatrix}.
    \]

    А т.\,к. на диагоналях $J_1$ и $J_2$ собственные значения одинаковые, то на диагонали матрицы
    $
    f(J) =
    \begin{pmatrix}
        f(J_1) & \\
         & f(J_2)
    \end{pmatrix}
    $ должны стоять одинаковые числа, а для $D$ это условие не выполняется.
\end{solution}

