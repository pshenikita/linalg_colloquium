\section{Жордановы клетки и матрицы, их характеристические и минимальные многочлены. Жорданов базис}

\begin{definition}
    Матрица вида
    $
    J_\lambda =
    \begin{pmatrix}
        \lambda & 1 & & 0\\
         & \lambda & \ddots & \\
         & & \ddots & 1 \\
        0 & & & \lambda
    \end{pmatrix}
    $ называется \textit{жордановой клеткой}. Если матрица оператора $\A$ в некотором базисе является блочно-диагональной с блоками вида $J_\lambda$ (возможно, соответствующих различным $\lambda$), то такая матрица называется \textit{жордановой нормальной формой} оператора $\A$. Базис, в котором оператор имеет жорданову нормальную форму, называется \textit{жордановым}.
\end{definition}

\begin{proposal}
    Минимальный аннулирующий многочлен оператора $\A$ над полем $\C$ есть $P(t) = \prod\limits_{i = 1}^k(t - \lambda_i)^{m_i}$, где $\lambda_1, \ldots, \lambda_k$ --- все собственные значения $\A$, а $m_i$ --- размер максимальной жордановой клетки, отвечающей $\lambda_i$.
\end{proposal}

\begin{proof}
    Мы имеет $R_{\lambda_i} = \Ker(\A - \lambda_i \cdot \id)^{m_i}$, поэтому многочлен $(t - \lambda_i)^{m_i}$ является минимальным аннулирующим для оператора $\A\big|_{R_{\lambda_i}}$. В силу теоремы о корневом разложении, $\forall v \in V$ представляется в виде $v = \sum\limits_iv_i$, где $v_i \in R_{\lambda_i}$. Т.\,к. $P(\A)$ содержит множитель $(\A - \lambda_i \cdot \id)^{m_i}$, мы имеем $P(\A)v_i = \bs{0}$, т.\,е. $P(\A)v = \bs{0}$, и многочлен $P(t)$ аннулирует оператор $\A$. С другой стороны, любой многочлен $Q(t)$, аннулирующий оператор $\A$, делится на минимальный аннулирующий многочлен для оператора $\A\big|_{R_{\lambda_i}}$, т.\,е. на $(t - \lambda_i)^{m_i}$, для каждого $\lambda_i$. Следовательно, $Q(t)$ делится на $P(t)$, и $P(t)$ --- минимальный многочлен.
\end{proof}

Про то, как искать жорданов базис, лучше всего \href{http://halgebra.math.msu.su/staff/klyachko/teaching/lin.al/JB1.PDF}{написал} Антон Александрович Клячко. Однако там этот алгоритм описан очень неподробно, когда-нибудь он появится здесь со всеми обоснованиями и доказательствами. Также напишу, как всё-таки искать количество жордановых базисов над конечным полем. То, что Антон Александрович записывает клетки с единицами под собственными значениями, а не над, как это принято, \textbf{существенно}. 

Как нетрудно заметить, у жордановой клетки минимальный многочлен совпадает с характеристическим. На самом деле, такие матрицы обладают особенным свойством.

\begin{problem}[А.\,А. Клячко]
    Докажите, что матрица коммутирует только с многочленами от себя тогда и только тогда, когда её минимальный многочлен совпадает с характеристическим.
\end{problem}

\begin{solution}
    Напишу позже. % TODO: написать решение
\end{solution}

