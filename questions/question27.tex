\section{Нормальный (канонический) вид квадратичной формы над полями действительных и комплексных чисел. Закон инерции}

Над полем $\R$ квадратичную форму можно далее упростить:

\begin{proposal}
    Для любой симметрической билинейной функции $\B$ в пространстве над полем $\R$ существует базис, в котором её матрица имеет диагональный вид с $1$, $-1$ и $0$ на диагонали.
\end{proposal}

\begin{proof}
    Сначала с помощью теоремы 1 приведём квадратичную форму к виду
    \[
        Q(u) = r_{11}(u^1)^2 + \ldots + r_{nn}(u^n)^2.
    \]

    Если $r_{ii} > 0$, то замена $y^i = \sqrt{r_{ii}}u^i$ приводит слагаемое $r_{ii}(u^i)^2$ к виду $(y^i)^2$. Если же $r_{ii} < 0$, то замена $y^i = \sqrt{\abs{r_{ii}}}u^i$ приводит слагаемое $r_{ii}(u^i)^2$ к виду $-(y^i)^2$. В результате получаем требуемый вид квадратичной формы с коэффициентами $1$, $-1$ и $0$.
\end{proof}

\begin{definition}
    Вид, описанный в последнем предложении, называется \textit{нормальным видом} вещественной симметрической билинейной формы (вещественной квадратичной формы).
\end{definition}

Над полем $\C$ квадратичную форму можно ещё больше упростить:

\begin{proposal}
    Для любой симметрической билинейной функции $\B$ над полем $\C$ существует базис, в котором её матрица имеет диагональный вид с $1$ и $0$ на диагонали.
\end{proposal}

\begin{proof}
    Сначала мы с помощью последнего предложения приведём квадратичную форму к виду $(y^1)^2 + \ldots + (y^p)^2 - (y^{p + 1})^2 - \ldots - (y^{p + q})^2$. Затем сделаем замену координат $y^k = z^k$ при $k \leqslant p$ и $y^k = iz^k$ при $k > p$. В результате получим требуемый вид, где $r = p + q = \rk Q$.
\end{proof}

\begin{definition}
    Вид, описанный в последнем предложении, называется \textit{нормальным видом} комплексной симметрической билинейной формы (комплексной квадратичной формы).
\end{definition}

В случае симметрической билинейной формы над $\C$ нормальный вид зависит только от её ранга, и поэтому мы получаем:

\begin{proposal}
    Две комплексные симметрические билинейные формы (комплексные квадратичные формы) получаются друг из друга линейной заменой координат только и только тогда, когда их ранги совпадают.
\end{proposal}

В случае симметрических билинейных форм над $\R$ ситуация сложнее: их нормальный вид не определяется одним лишь рангом, а зависит ещё от количества $1$ и $-1$ на диагонали матрицы. Оказывается, что нормальный вид такой формы не зависит от способа приведения к нормальному виду.

\begin{theorem}[Закон инерции]
    Количество $1$, $-1$ и $0$ на диагонали нормального вида матрицы вещественной симметрической билинейной функции $\B$ не зависит от способа приведения к нормальному виду.
\end{theorem}

\begin{remark}
    Важно понимать, что человечество на самом деле не умеет приводить квадратичные формы к какому-то адекватному виду. Мы хоть что-то знаем только про очень узкие ситуации --- симметричная (кососимметричная) матрица, только над полями $\R$ или $\C$ и т.\,п. Даже над полем $\Q$ понять, являются две квадратичные формы эквивалентными, сложно. Есть инвариант в виде ранга, есть замечание, что $\det A^\prime = \det (C^tAC) = (\det C)^2\det A$ (т.\,е. отношение определителей должно быть квадратом элемента поля). Но вот примерно на этом какие-то нормальные соображения заканчиваются. Ходят слухи, что в НМУ на <<Алгебре 2>> в 2024 году учили что-то понимать про поля типа $\Z_p$.
\end{remark}

