\section{Существование одномерного или двумерного инвариантного подпространства для любого линейного оператора над полем действительных чисел}

Оператор $\A: V \to V$ в нетривиальном пространстве над полем $\C$ имеет инвариантное подпространство размерности $1$. Действительно, т.\,к. поле $\C$ алгебраически замкнуто, характеристический многочлен $\chi_\A(t)$ имеет корень $\lambda$ и собственный вектор $v \in V_\lambda$. Подпространство $\langle v\rangle$ собственное размерности $1$.

Доказательство следующей теоремы использует понятие комплексификации линейного пространства. Про это написано в конспектах Панова и книге Винберга. Когда-нибудь и здесь появится такое приложение.

\begin{theorem}
    Оператор $\A: V \to V$ в нетривиальном пространстве над полем $\R$ имеет инвариантное подпространство размерности $1$ или $2$.
\end{theorem}

\begin{proof}
    Если характеристический многочлен $\chi_\A(t)$ имеет вещественный корень, то (аналогично) мы получаем одномерное инвариантное подпространство. Предположим, что $\chi_\A(t)$ не имеет вещественных корней. Пусть $\lambda + i\mu$ --- комплексный корень, $\mu \ne 0$. Тогда $\lambda + i\mu$ --- собственное значение комплексифицированного оператора $\A_\C$ (напомним, что в подходящих базисах матрицы операторов $\A$ и $\A_\C$ совпадают). Возьмём соответствующий собственный вектор $u + iv \in V_\C$. Тогда $\A u + i\A v = \A_\C(u + iv) = (\lambda + i\mu)(u + iv) = (\lambda u - \mu v) + i(\mu u + \lambda v)$. Следовательно, $\A u = \lambda u - \mu v$, а $\A v = \mu u + \lambda v$, и линейная оболочка $\langle u, v\rangle$ является инвариантным подпространством для $\A$.
\end{proof}

\begin{corollary}
    Над полем $\R$ любой линейный оператор приводим к блочно-диагональному виду, причём блоки имеют порядок не выше двух.
\end{corollary}

